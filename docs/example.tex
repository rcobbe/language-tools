\documentclass[twoside]{article}

%% More up-to-date example output is in ~/languages/greek/hq/lexicon.pdf.

\usepackage{fontspec}
\usepackage{xltxtra}
\usepackage{xunicode}
\usepackage[margin=1in]{geometry}
\usepackage{hanging}
\usepackage{multicol}
\usepackage{polyglossia}

%% due to a bug in TeXLive 2010, specifying usmax variant produces bogus
%% output before the first lexicon entry.
\setdefaultlanguage{english}%[variant=usmax]{english}
\setromanfont[Mapping=tex-text]{Constantia}

\setotherlanguages{greek,latin,sanskrit}
\setkeys{greek}{variant=ancient,numerals=arabic}
\newfontfamily\greekfont{Gentium Plus}[CharacterVariant=78]

%% \newfontfamily\sanskritfont[Script=Devanagari]{Sanskrit 2003}
\newfontfamily\sanskritfont{Devanagari MT}

\newcommand{\gk}{\textgreek}
\newcommand{\lt}{\textlatin}
\newcommand{\sk}{\textsanskrit}

\newenvironment{entry}[1]{%
  \noindent\hangpara{1em}{1}\textbf{#1}\markboth{#1}{#1}}{%
  \raggedright\par}

%% two environments for tyepsetting subentries.  The first treats it as a
%% full entry, merely indented within the containing entry.  The second
%% typesets it "inline" in the larger entry.
%%
%% XXX: For the first variant, is there a way to repeat the main entry's
%% head word if there's a page or column break right before the subentry?
%% And do we need similar repeated headings when the second variant occurs
%% at a page break?
% \newenvironment{subentry}[1]{%
%   \begin{list}{\mbox{}}{%
%       \setlength{\topsep}{0ex}%
%       \setlength{\leftmargin}{2.5em}%
%       \setlength{\itemindent}{-1em}%
%     }%
%   \item\textbf{#1}}{%
%     \raggedright\end{list}}
\newenvironment{subentry}[1]{\hspace{.5em}\textbf{#1}}{}

%% subcategorization information
\newcommand{\subcat}[1]{\textit{#1}}

%% syntactic information
\newcommand{\syn}[1]{\textit{#1}}

%% gender
\newcommand{\gender}[1]{\textit{#1}}

%% citation of usage
%%  1: original quote
%%  2: english
%%  3: source info
\newcommand{\example}[3]{\textit{#1 (#2), #3}}

\begin{document}
\pagestyle{myheadings}
\begin{multicols}{2}

\begin{entry}{\lt{amō}}
  -āre, to like, love (W~1)
\end{entry}

\begin{entry}{cōgitō} -āre, to
think, ponder, consider, plan (W~1).  Derivatives:
cogitate.
\end{entry}

\begin{entry}{cōnservō} -āre (stronger form of servō), to
  preserve, conserve, maintain (W~1).  Derivatives:
  conservative, conservation.
\end{entry}

\begin{entry}{dēbeō} dēbēre dēbuī dēbitum (< dehibeō < de-habeō, to have
  from a person), to owe; ought, must (W~1).  Derivatives:
  debt, debit.
\end{entry}

\begin{entry}{dō} dare dedī datum (2sg pres. act. indic. dās,
  2sg. pres. act. imper. dā), to give, offer (W~1).
  Derivatives: date, data.
\end{entry}

\begin{entry}{fāma} -ae, \gender{f.}, 1.~rumor, report. 2.~fame, reputation (W~2).
  Derivatives: famous, defame, infamy.
\end{entry}

\begin{entry}{fōrma} -ae, \gender{f.}, 1.~form, shape. 2.~beauty (W~2).
  Derivatives: formal, format, formula, formless, deform, inform, etc., but
  not formic, formidable.
\end{entry}

\begin{entry}{fortūna} -ae, \gender{f.}, fortune, luck (W~2).
\end{entry}

\begin{entry}{īra} -ae, \gender{f.}, ire, anger (W~1).  Derivatives: irate,
  irascible, but not irritate.
\end{entry}

\begin{entry}{laudō} -āre, to praise (W~1).  Derivatives:
  laud, laudable, laudatory.
\end{entry}

\begin{entry}{moneō} -ēre -uī -itum, to remind, advise, warn (W~1);
  \example{\lt{monē mē sī errō}}{warn me if I go astray}{Virgil, Aeneid, I.243}.
  Derivatives: admonish, admonition, monitor, monument, monster.
\end{entry}

\begin{entry}{nauta} -ae, \gender{m.}, sailor (W~2).
\end{entry}

\begin{entry}{patria} -ae, \gender{f.}, fatherland, native land, (one's) country
  (W~2).\end{entry}

\begin{entry}{pecūnia} -ae, \gender{f.}, money (W~2).  Derivatives: pecuniary,
  impecunious; cp. peculation.
\end{entry}

\begin{entry}{philosophia} -ae, \gender{f.} (< \gk{φιλοσοφί\underline{α}}),
  philosophy (W~2).
\end{entry}

\begin{entry}{poena} -ae, \gender{f.}, penalty, punishment (W~2).  Pad the
  length out to see what subentries look like in a slightly more
  interesting context.
\begin{subentry}{poenās dare}, to pay the penalty (W~2).  Pad the length
  out to see what this looks like with multiple lines.  Two lines are
  necessary; three would be even better.  Four is overkill.  Five is right
  out.
  \end{subentry}
\end{entry}

\begin{entry}{poēta} -ae, \gender{m.}, poet (W~2).\end{entry}

\begin{entry}{porta} -ae, \gender{f.}, gate, entrance\end{entry}

\begin{entry}{salveō} -ēre --- ---, 1.~to be well, be in good health.
  2.~pres. imper.: greetings (W~1).  Derivatives:
  salvation, salve.
\end{entry}

\begin{entry}{servō} -āre, to preserve, save, keep, guard
  (W~1).  Derivatives: reserve, reservoir.
\end{entry}

\begin{entry}{sine} \subcat{+ abl.} without.\end{entry}

\begin{entry}{terreō} -ēre -uī -itum, to frighten, terrify
  (W~1).  Derivatives: terrible, terrific, terror, deter.
\end{entry}

\begin{entry}{valeō} -ēre -uī valitūrum, to be strong, have power; to be
  well; pres imper.: good-bye, farewell (W~1).
  Derivatives: valid, invalidate, prevail, prevalent, valedictory.
\end{entry}

\begin{entry}{videō} -ēre vīdī vīsum, to see, observe; understand
  (W~1).  Derivatives: provide, evident, view, review.
\end{entry}

\begin{entry}{vocō} -āre, to call, summon (W~1).
  Derivatives: vocation, advocate, vocabulary, convoke, evoke, invoke,
  provoke, revoke.
\end{entry}

\begin{entry}{\gk{ἀγορ\underline{ά}}} \gk{-ᾶς}, \gender{f.}, marketplace, civic center,
  cultural center (HQ~1).  Derivatives: agoraphobia.
\end{entry}

\begin{entry}{\gk{ἀδελφός}} \gk{-οῦ} (voc. sg. \gk{ἄδελφε}), \gender{m.},
  brother\end{entry}

\begin{entry}{\gk{ἄνθρωπος}} \gk{-ου}, \gender{m.}, man, human being.\end{entry}

\begin{entry}{\gk{ἀπό}}, \subcat{+ gen.} from, away from
\end{entry}

\begin{entry}{\gk{βιβλίον}} \gk{-ου}, \gender{n.}, book.\end{entry}

\begin{entry}{\gk{δῶρον}} \gk{-ου}, \gender{n.}, gift; bribe (esp. pl).
\end{entry}

\begin{entry}{\gk{καί \ldots{} καί}}, both ... and
\end{entry}

\begin{entry}{\gk{μέν \ldots{} δέ}} \syn{postpos conj}, on one hand \ldots{}
  on the other
\end{entry}

\begin{entry}{\gk{οὐ}} (\gk{οὐκ} before smooth breathing, \gk{οὐχ} before
  rough), not
\end{entry}

\begin{entry}{\gk{παρά}}, 1.~\subcat{+ gen,} from (the side of); 2.~\subcat{+
    dat,} at (the side of); 3.~\subcat{+ acc,} to (the side of); beside;
  4.~\subcat{+ dat,} at the house of; 5.~\subcat{+ acc,} contrary to.
\end{entry}

\begin{entry}{\gk{πέμπω}} \gk{πέμψω ἔπεμψα πέπομφα πέπεμμαι ἐπέμφθην}, to
  send (HQ~2).
\end{entry}

\begin{entry}{\sk{कृष्}} \sk{कर्षति} 1, to draw (SM~I.2).
\end{entry}

\end{multicols}
\end{document}

%%% Local Variables:
%%% TeX-command-default: "XeLaTeX"
%%% End:
